\section{Method - Hierarchical Clustering}
In this final project, our group clustered the data into five segment and use hierarchical clustering method. Hierarchical Clustering is a method of cluster analysis which seeks to build a hierarchy of clusters. Strategies for hierarchical clustering generally fall into two types:
\begin{itemize}
	\item \textbf{Agglomerative}, This is a "bottom up" approach: each observation starts in its own cluster, and pairs of clusters are merged as one moves up the hierarchy.
	\item \textbf{Divisive}, This is a "top down" approach: all observations start in one cluster, and splits are performed recursively as one moves down the hierarchy.
\end{itemize}

In general, the merges and splits are determined in a greedy manner. The results of hierarchical clustering are usually presented in a dendrogram.
